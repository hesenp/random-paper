\documentclass{sig-alternate}

\begin{document}


\title{Probablistic Association Discovery using Functions of
  Observation Graph} 
\numberofauthors{2}

\author{
  \alignauthor 
  Hesen Peng \\
  \affaddr{Independent}\\
  \affaddr{136 102nd Ave SE Apt 326}\\
  \affaddr{Bellevue, Washington, USA 98004}\\
  \email{hesen.peng@gmail.com}  
  \alignauthor
  Tianwei Yu\\
  \affaddr{Dept of Biostatistics and Bioinformatics}\\
  \affaddr{Emory University}\\
  \affaddr{1518 Clifton Rd NE 3F}\\
  \affaddr{Atlanta Georgia, USA 30322}\\
  \email{tianwei.yu@emory.edu}
}

\date{\today{}}

\maketitle
 

\begin{abstract}
  Probablistic association discovery aims to identify the association
  between random vectors of their association, regardless of number of
  variables involved or linear/nonlinear functional forms. Application
  to high-dimensional data analysis has generated rising interest in
  probablistic association discovery.

  We developed a framework for probablistic association discovery on
  the Euclidean space using functions on the observation graph. We
  first discuss the property of mean observation distance and its
  novel application to association discovery. Then we generalize the
  advantageous property to a group of functions on the observation
  graph. The group of functions encapsulates major existing methods in
  association discovery, like mutual information and Mira score. We
  conducted numerical comparison that shows differentiated testing
  power under multiple scenarios. 

\end{abstract}


\category{H.4}{Information Systems Applications}{Miscellaneous}
\category{D.2.8}{Software Engineering}{Metrics}[complexity measures, performance measures]

\terms{Theory, Happy}

\keywords{ACM proceedings, \LaTeX, text tagging}


\section{Introduction}
\label{sec:intro}

In this paper we would like to discuss emerging methods on the
discovery of universal probablistic assocation between random vectors.
Consider two random vectors $X$ and $Y$ and $n$ pairs of independent
and indentically distributed (i.i.d.) random samples $\{X_i,
Y_i\}_{i=1}^n$. We would like to draw inference for the existence
between $X$ and $Y$ based on the $n$ pairs of samples. Classical
association statistics like Pearson's correlation coefficient assume
functional forms (linear, monotonicity) between $X$ and $Y$, which are
judged as \emph{uncorrelated} if
\begin{displaymath}
  Corr(X,Y)=0
\end{displaymath}
Universal association statistic perceive associations from the level
of probablistic dependence. That is, $X$ and $Y$ are judged as
independent if and only if
\begin{equation}
  \label{eq:independence}
  F(X,Y)= F(X) F(Y)
\end{equation}
where $F(\cdot)$ is the probability density function for the random
vector under consideration. Probabblistic association as captured by
universal association statistics encapsulates a larger group of
associations than traditional correlation coefficient. For example,
universal association would consider nonlinear interactions involving
multiple variables. 

We have noticed that multitude of methods on universal association
discovery link to distance functions on the observation graph. The
distance graph consists of nodes representing each observation $(X_i,
Y_i)$ in the $p+q$ Euclidean space. Here $p$ and $q$ are the
dimensions of $X$ and $Y$, respectively. Edges of the observation
graph would connect two nodes (observations) if specific criteria is
satisfied.

For example, mutual information and its derivatives have been the most
popular universal association statistic to date\cite{cite:my-cas-work,
  citation:MINE-science, tostevin2009mutual}. To estimate mutual
information, the joint entropy can be approximated using $K$-nearest
neighbour distance \cite{PhysRevE.69.066138, leonenko2008,
  doi:10.1080/104852504200026815}. Recent breakthrough on distance
covariate \cite{székely2009, székely2007} sheds light on universal
association discovery with its simplicity of form and theoretical
flexibility. Brownian distance \cite{székely2009} covariate proposed
\texttt{dCov} as
\begin{equation*}
  V_N^2 = \frac{1}{n^2}\sum_{k,l=1}^n D^{X}_{kl}D^{Y}_{kl}
\end{equation*}
where $D^{X}_{kl}$ and $D^{Y}_{kl}$ are simple linear functions of
pairwise distances between sample elements calculated on $X$ and $Y$
dimensions, respectively. In an independent research, the author
proposed Mira score \cite{my-dissertation} as 
\begin{equation*}
  M = \sum_{k,l = 1}^n D^{(X,Y)}_{kl} w_{kl}
\end{equation*}
where $D^{(X,Y)}_{kl}$ is the distance between sample elements
calculated using both $X$ and $Y$ dimensions, and $w_{kl}=1$ when the
involved elements are nearest neighbors, $w_{kl}=0$ otherwise.

In this article we will contribute: 
\begin{enumerate}
\item Point out that non-trial functions of the observation graph
  would be capable of discoverying universal association.
\item Present numerical comparison between existing methods.
\end{enumerate}

\section{Functions on the Observation Graph}
\label{sec:funcs}



\bibliographystyle{abbrv}
\bibliography{mira-score}


\end{document}
